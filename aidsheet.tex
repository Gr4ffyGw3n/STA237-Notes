\documentclass[5pt]{article}
\usepackage{amsmath, amsthm, amssymb, amsfonts}
\usepackage{thmtools}
\usepackage{graphicx}
\usepackage{setspace}
\usepackage{geometry}
\usepackage{float}\usepackage{hyperref}
\usepackage[utf8]{inputenc}
\usepackage[english]{babel}
\usepackage{framed}
\usepackage[dvipsnames]{xcolor}
\usepackage{tcolorbox}
\usepackage{outlines}
\newcounter{Chapcounter}
\usepackage[compact]{titlesec}


\newcommand\showmycounter{\addtocounter{Chapcounter}{1}\themycounter}
\newcommand{\chapter}[1] 
{ {\centering          
  \addtocounter{Chapcounter}{1} \Large \underline{\textbf{ \color{blue} Chapter \theChapcounter: ~#1}} }   
  \addcontentsline{toc}{section}{ \color{blue} Chapter:~\theChapcounter~~ #1}    
}

\colorlet{LightGray}{White!90!Periwinkle}
\colorlet{LightOrange}{Orange!15}
\colorlet{LightGreen}{Green!15}
\colorlet{LightLavender}{Lavender!15}
\newcommand{\HRule}[1]{\rule{\linewidth}{#1}}

\declaretheoremstyle[name=Theorem,]{thmsty}
\declaretheorem[style=thmsty,numberwithin=section]{theorem}
\tcolorboxenvironment{theorem}{colback=LightGray}

\declaretheoremstyle[name=Proposition,]{prosty}
\declaretheorem[style=prosty,numberlike=theorem]{proposition}
\tcolorboxenvironment{proposition}{colback=LightOrange}

\declaretheoremstyle[name=Principle,]{prcpsty}
\declaretheorem[style=prcpsty,numberlike=theorem]{principle}
\tcolorboxenvironment{principle}{colback=LightGreen}

\declaretheoremstyle[name=Example,]{exsty}
\declaretheorem[style=exsty,numberlike=theorem]{example}

\declaretheoremstyle[name=Definition,]{defsty}
\declaretheorem[style=defsty,numberlike=theorem]{definition}

\setstretch{1.2}
\geometry{
    textheight=10.5in,
    textwidth=8in,
    top=0.25in,
    headheight=1pt,
    headsep=1pt,
    footskip=30pt
}

% ------------------------------------------------------------------------------

\begin{document}
\section{Outcomes, events, and Probability}
\begin{definition}
\textbf{(Random) Experiment} is a mechanism that results in random outcomes
\end{definition}

\begin{definition}
\textbf{Sample Space} ($\Omega$) is the set of all possible outcomes from and experiment
\end{definition}

\begin{definition}
\textbf{Event} is a subset of $\Omega$
\end{definition}

\begin{definition}
\textbf{Mutually Exclusive} (Disjoint): $A\cup B = \{\} =\varnothing$
\end{definition}

\begin{definition}
\textbf{Commutative}: $A\cup B = B \cup A$ $\vert$ \textbf{Associative}: $(A\cup B) \cup C = A \cup (B \cup C)$ $\vert$ \\
\textbf{Distributive} $A\cup (B\cap C) = (A\cup B)\cap (A\cup C)$, $A\cap (B\cup C) = (A\cap B)\cup (A\cap C)$
\end{definition}

\begin{definition}
\textbf{A implies B} $A\cap B = A$, $A\subset B$
\end{definition}

\begin{definition}
\textbf{Probability Function} $P$ on finite $\Omega$ assigns each event $A$ a $P(A)$ s.t. (Axioms) \\
i) $P(A) \geq 0$ ii)$P(\Omega) = 1$ iii)$P(A\cup B) = P(A)+P(B)$ if disjoint.
\end{definition}

\begin{definition}
\textbf{Useful Probabilities}: \emph{Union}: $P(A\cup B) = P(A\cap B^c) + P(A\cap B) + P(A^c \cap B)$, $P(A\cup B) = P(A) + P(B) - P(A\cap B)$ \\
\emph{Complement} $P(\Omega) = P(A) + P(A^c)$, $P(A^c) = 1 - P(A)$
\end{definition}

\begin{definition}
\textbf{Calculating by counting} applies only when i) all outcomes are equally likely ii) $\Omega$ is finite then \\
$P(A) = \frac{\text{number of outcomes belonging to }A}{\text{total number of outcomes in }\Omega}$
\end{definition}

\begin{definition}
\textbf{Product of Sample Space} in general is $\Omega = \Omega_1 \times \Omega_2 = \{(\omega_1, \omega_2): \omega_1 \in \Omega_1, \omega_2 \in \Omega_2\}$
\end{definition}

\begin{definition}
\textbf{Permutation} of a set \textbf{\emph{N}} of size \textbf{\emph{n}} is $_NP_n = \frac{N!}{(N-n)!}$.
\end{definition}

\begin{definition}
\textbf{Combination} of a set \textbf{\emph{N}} of size \textbf{\emph{n}} is ${N \choose n} = \frac{N!}{(N-n)! \cdot N!}$
\end{definition}

\begin{definition}
\textbf{De Morgan's Law} for $A, B$, we have $(A\cup B)^c = A^c \cap B^c$ and $(A\cap B)^c = A^c \cup B^c$
\end{definition}


\section{Conditional Probability}
\begin{definition}
\textbf{Conditional Probability} of $A$ given $C$ is, in general, $P(A|C) = \frac{P(A\cap C)}{C}$ for any $P(C)>0$
\end{definition}

\begin{definition}
\textbf{Law of Total Probability} states, for disjoint $C_1,\dots,C_m$, $C_1 \cup \dots \cup C_m = \Omega$, $P(A) = \sum^m_{i=1} [P(A|C_i)P(C_i)]$
\end{definition}

\begin{definition}
\textbf{Bayes' Rule} states, for disjoint $C_1,\dots,C_m$, $C_1 \cup \dots \cup C_m=\Omega$, $P_(C_i|A) = \frac{P(A|C_i)\cdot P(C_i)}{\sum^m_{i=1}[P(A|C_i)P(C_i)}$
\end{definition}

\begin{definition}
\textbf{Independence} between $A$ and $B$ implies $P(A|B) = P(A)$
\end{definition} 

\begin{definition}
\textbf{Relation between Independent Probabilities}: i) $A \text{\emph{ independent of}} B \iff A^c \text{\emph{ independent of }}B$ and \\
ii) $A \text{\emph{ independent of }}B \iff B\text{\emph{ independent of }}A$
\end{definition}

\begin{definition}
\textbf{Independence of Two or More Events}: $P(A_1\cap A_2 \cap \dots \cap A_n) = \prod^n_{i=1}A_i$
\end{definition}

\section{Random Variables}
\begin{definition}
The \textbf{probability mass function} of drv \textbf{X} is $p: \mathbb{R} \to [0,1]$ defined by $p(k) = P(x=k)$ for $-\infty < k < \infty$.
\end{definition}

\begin{definition}
\textbf{Continuous Random Variable} $X$ has $P(a\leq X \leq B) = \int^b_a f(x) dx$ with i) $\forall x, f(x) > 0$ ii) $\int^\infty_{-\infty}f(x)dx = 1$.\\
$f$ is the \textbf{probability density function} of $X$ and $f(x)$ is the \textbf{probability density} of $X$ at $x$. Note: a random variable is $continous$ if its cdf is continuous everywhere
\end{definition}

\begin{definition}
The \textbf{cumulative distribution function} $F$ of a drv or crv \textbf{X} is the function $F: \mathbb{R}\to [0,1]$ defined by $F(a) = P(X \leq a)$ for $-\infty < a < \infty$
\end{definition}

\section{Common Distributions}
\subsection{Discrete}
\begin{definition}
\textbf{Bernoulli} $X \sim \textbf{Ber}(\theta)$: Parameter $\theta$, $0 \leq \theta \leq 1$ and \textbf{pmf} given by
$
p_X(X)=
\begin{cases}
\theta &\qquad x = 1\\
1 - \theta &\qquad x = 0\\ 
\end{cases}$\\
Useful for modelling experiments with exactly two possible outcomes
\end{definition}

\begin{definition}
\textbf{Binomial} $\text{Bin}(n,\theta)$: Parameters $n$ and $\theta$ with $n \in \mathbb{N}$ and $0 \leq \theta \leq 1$ and \textbf{pmf} given by $p_X(x) = {n \choose x}\theta^x(1-\theta)^{n-x}$\\
This distribution describes a sum of $n$ \emph{\textbf{independent}} and \emph{\textbf{identical}} Bernoulli trials
\end{definition}

\begin{definition}
\textbf{Geometric} $\text{Geo}(\theta)$: Parameters $\theta$ with $0 \leq \theta \leq 1$ and \textbf{pmf} given by $p_X(x) = (1-\theta)^{x-1} \theta$ for $x \in \mathbb{N}.$\\
The number of identical Bernoulli trials until the first event
\end{definition}

\begin{definition} 
\textbf{Poisson} $\text{Pois}(\lambda)$: Parameters $\lambda$ with $\lambda > 0$ and \textbf{pmf} given by $p_X(x) = \frac{e^{-\lambda}\lambda^x}{x!}$ for $x \in \mathbb{Z}^+$\\
Captures the count of events in a fixed interval of Poisson Process. Assumptions: i) the expected rate $\lambda$ is constant ii) all events are independent of each other iii) events cannot occur simultaneously
\end{definition}

\subsection{Continuous}
\begin{definition}
\textbf{Uniform} $U(\alpha,\beta)$: on interval $[\alpha,\beta]$ and \textbf{pdf} given by $f(x) = \begin{cases}
\frac{1}{\beta - \alpha} \qquad & \alpha \leq x \leq \beta\\
0 &\textbf{otherwise}
\end{cases}$\\
Uniform distribution assigns equal probabilities across a fixed interval, useful for modelling \emph{completely arbitrary} experiments/\emph{complete ignorance} about probabilities
\end{definition}

\begin{definition}
\textbf{Exponential} $\textbf{Exp}(\lambda)$: Parameter $\lambda$ with $\lambda > 0$ and \textbf{pdf} $f(x) = \begin{cases}
\lambda e^{-\lambda x} \qquad & x \geq 0\\
0 &otherise
\end{cases}$\\
Useful for modelling time until next event in a Poisson process. $\lambda$ is the expected rate of events. Note: $F_Y$ is continuous everywhere, $f_Y$ is discontinuous at 0.
\end{definition}

\begin{definition}
\textbf{Gamma} $G \sim \textbf{Gamma}(\alpha, \beta)$: Parameters $\alpha, \beta$ with $\beta, \alpha > 0$ with \textbf{pdf} given by $f(x)=\frac{1}{\Gamma(a)}\beta^\alpha x^{\alpha - 1} e^{-\beta x}$ for $x>0$\\
\end{definition}

\begin{definition}
\textbf{Normal} $N(\mu, \sigma^2)$: Parameters $\mu, \sigma^2$ with $\sigma^0 > 0$ with \textbf{pdf} $f(x)=\frac{1}{\sigma\sqrt{2\pi}}\exp{\{-\frac{1}{2}(\frac{x-\mu}{\sigma}^2)\}}$\\
\textbf{Standard Normal Distribution} is when $\mu = 0$ and $\sigma^2 = 1$ with \textbf{pdf} $\phi = \frac{1}{\sqrt{2pi}}e^{-\frac{1}{2}z^2}$ and \textbf{cdf} $\Phi = \int^a_{-\infty}\frac{1}{\sqrt{2\pi}}e^{-\frac{1}{2}z^2}dz$\\
Often used to model observational errors. Note: Normal distributions have symmetry along its centre, $\mu$ and $\sigma$ controls the spread of distribution.
\end{definition}

\section{Quantile, Percent, and Median}
\begin{definition}
\textbf{Quantile Function} of random variable $X$ with \textbf{cdf} $F$ is $F^{-1}=\min{\{x: F(x) \geq T\}}$ for $0\leq T \leq 1$
\end{definition}
\section{Useful Stuff}
\begin{equation}
{n \choose k} = \frac{n!}{k!(n-k)!}
\end{equation}
\begin{equation}
x=\frac{-b\pm\sqrt{b^2-4ac}}{2a}
\end{equation}


\end{document}