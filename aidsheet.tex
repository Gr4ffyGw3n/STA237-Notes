\documentclass[5pt]{article}
\usepackage{amsmath, amsthm, amssymb, amsfonts}
\usepackage{thmtools}
\usepackage{graphicx}
\usepackage{setspace}
\usepackage{geometry}
\usepackage{float}\usepackage{hyperref}
\usepackage[utf8]{inputenc}
\usepackage[english]{babel}
\usepackage{framed}
\usepackage[dvipsnames]{xcolor}
\usepackage{tcolorbox}
\usepackage{outlines}
\newcounter{Chapcounter}


\newcommand\showmycounter{\addtocounter{Chapcounter}{1}\themycounter}
\newcommand{\chapter}[1] 
{ {\centering          
  \addtocounter{Chapcounter}{1} \Large \underline{\textbf{ \color{blue} Chapter \theChapcounter: ~#1}} }   
  \addcontentsline{toc}{section}{ \color{blue} Chapter:~\theChapcounter~~ #1}    
}

\colorlet{LightGray}{White!90!Periwinkle}
\colorlet{LightOrange}{Orange!15}
\colorlet{LightGreen}{Green!15}
\colorlet{LightLavender}{Lavender!15}
\newcommand{\HRule}[1]{\rule{\linewidth}{#1}}

\declaretheoremstyle[name=Theorem,]{thmsty}
\declaretheorem[style=thmsty,numberwithin=section]{theorem}
\tcolorboxenvironment{theorem}{colback=LightGray}

\declaretheoremstyle[name=Proposition,]{prosty}
\declaretheorem[style=prosty,numberlike=theorem]{proposition}
\tcolorboxenvironment{proposition}{colback=LightOrange}

\declaretheoremstyle[name=Principle,]{prcpsty}
\declaretheorem[style=prcpsty,numberlike=theorem]{principle}
\tcolorboxenvironment{principle}{colback=LightGreen}

\declaretheoremstyle[name=Example,]{exsty}
\declaretheorem[style=exsty,numberlike=theorem]{example}
\declaretheoremstyle[name=Definition,]{defsty}
\declaretheorem[style=defsty,numberlike=theorem]{definition}
\setstretch{1.2}
\geometry{
    textheight=14in,
    textwidth=8in,
    top=1in,
    headheight=5pt,
    headsep=5pt,
    footskip=30pt
}

% ------------------------------------------------------------------------------

\begin{document}
\chapter{Outcomes, Probability}

\chapter{Random Variables}
\section{General Definitions}
\begin{definition}
\textbf{(Random) Experiment} is a mechanism that results in random outcomes
\end{definition}

\begin{definition}
\textbf{Sample Space} ($\Omega$) is the set of all possible outcomes from and experiment
\end{definition}

\begin{definition}
\textbf{Mutually Exclusive} (Disjoint): $A\cup B = {} =\varnothing$
\end{definition}

\begin{definition}
\textbf{A implies B} $A\cap B = A$, $A\subset B$
\end{definition}

\begin{definition}
\textbf{Probability Function} $P$ on finite $\Omega$ assigns each event $A$ a $P(A)$ s.t. i) $P(A) \geq 0$ ii)$P(\Omega) = 1$ iii)$P(A\cup B) = P(A)+P(B)$ if disjoint.
\end{definition}

\begin{definition}
\textbf{De Morgan's Law} for $A, B$, we have $(A\cup B)^c = A^c \cap B^c$ and $(A\cap B)^c = A^c \cup B^c$
\end{definition}



\begin{definition}
\textbf{Event} is a subset of $\Omega$
\end{definition}

\begin{definition}
The \textbf{probability mass function} of drm \textbf{X} is $p: \mathbb{R} \to [0,1]$ defined by $p(k) = P(x=k)$ for $-\infty < k < \infty$.
\end{definition}

\begin{definition}
The \textbf{cumulative distributiotion fuction} $F$ of a drm \textbf{X} is the function $F: \mathbb{R}\to [0,1]$ defined by $F(a) = P(X \leq a)$ for $-\infty < a < \infty$
\end{definition}

\section{Common Distributions}
\subsection{Discrete}
\begin{definition}
\textbf{Bernoulli} $X \sim \textbf{Ber}(\theta)$: Parameter $\theta$, $0 \leq \theta \leq 1$ and \textbf{pmf} given by
$
p_X(X)=
\begin{cases}
\theta &\qquad x = 1\\
1 - \theta &\qquad x = 0\\ 
\end{cases}$
\end{definition}

\begin{definition}
\textbf{Binomial} $\text{Bin}(n,\theta)$: Parameters $n$ and $\theta$ with $n \in \mathbb{N}$ and $0 \leq \theta \leq 1$ and \textbf{pmf} given by $p_X(x) = {n \choose x}\theta^x(1-\theta)^{n-x}$
\end{definition}

\begin{definition}
\textbf{Geometric} $\text{Geo}(\theta)$: Parameters $\theta$ with $0 \leq \theta \leq 1$ and \textbf{pmf} given by $p_X(x) = (1-\theta)^{x-1} \theta$ for $x \in \mathbb{N}.$
\end{definition}

\begin{definition} 
\textbf{Poisson} $\text{Pois}(\lambda)$: Parameters $\lambda$ with $\lambda > 0$ and \textbf{pmf} given by $p_X(x) = \frac{e^{-\lambda}\lambda^x}{x!}$ for $x \in \mathbb{Z}^+$
\end{definition}
\subsection{Continuous}
\begin{definition}
\textbf{Uniform} $U(\alpha,\beta)$: on interval $[\alpha,\beta]$ and \textbf{pdf} given by $f(x) = \begin{cases}
\frac{1}{\beta - \alpha} \qquad & \alpha \leq x \leq \beta\\
0 &\textbf{otherwise}
\end{cases}$
\end{definition}

\begin{definition}
\textbf{Exponential} $\textbf{Exp}(\lambda)$: Parameter $\lambda$ with $\lambda > 0$ and \textbf{pdf} $f(x) = \begin{cases}
\lambda e^{-\lambda x} \qquad & x \geq 0\\
0 &otherise
\end{cases}$
\end{definition}

\begin{definition}
\textbf{Gamma} $G \sim \textbf{Gamma}(\alpha, \beta)$: Parameters $\alpha, \beta$ with $\beta, \alpha > 0$ with \textbf{pdf} given by $f(x)=\frac{1}{\Gamma(a)}\beta^\alpha x^{\alpha - 1} e^{-\beta x}$ for $x>0$
\end{definition}

\begin{definition}
\textbf{Normal} $N(\mu, \sigma^2)$: Parameters $\mu, \sigma^2$ with $\sigma^0 > 0$ with \textbf{pdf} $f(x)=\frac{1}{\sigma\sqrt{2\pi}}\exp{\{-\frac{1}{2}(\frac{x-\mu}{\sigma}^2)\}}$\\
\textbf{Standard Normal Distribution} is when $\mu = 0$ and $\sigma^2 = 1$ with \textbf{pdf} $\phi = \frac{1}{\sqrt{2pi}}e^{-\frac{1}{2}z^2}$ and \textbf{cdf} $\Phi = \int^a_{-\infty}\frac{1}{\sqrt{2\pi}}e^{-\frac{1}{2}z^2}dz$
\end{definition}

\chapter{Quantile, Percent, and Median}
\begin{definition}
\textbf{Quantile Function} of random variable $X$ with \textbf{cdf} $F$ is $F^{-1}=\min{\{x: F(x) \geq T\}}$ for $0\leq T \leq 1$
\end{definition}
\end{document}