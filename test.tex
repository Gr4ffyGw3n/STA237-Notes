\documentclass{article}
\usepackage{amsmath, amsthm, amssymb, amsfonts}
\usepackage{thmtools}
\usepackage{graphicx}
\usepackage{setspace}
\usepackage{geometry}
\usepackage{float}\usepackage{hyperref}
\usepackage[utf8]{inputenc}
\usepackage[english]{babel}
\usepackage{framed}
\usepackage[dvipsnames]{xcolor}
\usepackage{tcolorbox}
\usepackage{outlines}
\newcounter{Chapcounter}


\newcommand\showmycounter{\addtocounter{Chapcounter}{1}\themycounter}
\newcommand{\chapter}[1] 
{ {\centering          
  \addtocounter{Chapcounter}{1} \Large \underline{\textbf{ \color{blue} Chapter \theChapcounter: ~#1}} }   
  \addcontentsline{toc}{section}{ \color{blue} Chapter:~\theChapcounter~~ #1}    
}

\colorlet{LightGray}{White!90!Periwinkle}
\colorlet{LightOrange}{Orange!15}
\colorlet{LightGreen}{Green!15}
\colorlet{LightLavender}{Lavender!15}
\newcommand{\HRule}[1]{\rule{\linewidth}{#1}}

\declaretheoremstyle[name=Theorem,]{thmsty}
\declaretheorem[style=thmsty,numberwithin=section]{theorem}
\tcolorboxenvironment{theorem}{colback=LightGray}

\declaretheoremstyle[name=Proposition,]{prosty}
\declaretheorem[style=prosty,numberlike=theorem]{proposition}
\tcolorboxenvironment{proposition}{colback=LightOrange}

\declaretheoremstyle[name=Principle,]{prcpsty}
\declaretheorem[style=prcpsty,numberlike=theorem]{principle}
\tcolorboxenvironment{principle}{colback=LightGreen}

\declaretheoremstyle[name=Example,]{exsty}
\declaretheorem[style=exsty,numberlike=theorem]{example}
\declaretheoremstyle[name=Definition,]{defsty}
\declaretheorem[style=defsty,numberlike=theorem]{definition}
\setstretch{1.2}
\geometry{
    textheight=9in,
    textwidth=5.5in,
    top=1in,
    headheight=12pt,
    headsep=25pt,
    footskip=30pt
}

% ------------------------------------------------------------------------------

\begin{document}

% ------------------------------------------------------------------------------
% Cover Page and ToC
% ------------------------------------------------------------------------------

\title{ \normalsize \textsc{}
		\\ [2.0cm]
		\HRule{1.5pt} \\
		\LARGE \textbf{\uppercase{STA 237 Notes} \vspace*{10\baselineskip}}
		}
\date{}
\author{\textbf{Author} \\ 
		Hao Hua He \\}

\maketitle
\newpage

\tableofcontents
\newpage

% -----------
\chapter{Week 1}
\section{Lec 1: Outcomes, Events. and Probability}
\subsection{Introduction}
Definitions
\begin{outline}[enumerate]
\1 \textbf{Probability}
\2 numeric value of certainty/uncertainty
\1 \textbf{(Random) Experiment}
\2 mechanism/phenomenon that results in random or unpredictable outcomes
\1 \textbf{Sample Space}
\2 Set of all outcomes from an experiment
\2 denoted $\Omega$
\1 \textbf{Event}
\2 Subset of Sample Space
\2 \underline{Relations between events}
\3 INtersect
\3 Union
\3 Complement
\end{outline}

\begin{example}
Neither $A$ not $B$ is denoted $(A\cup B)^c \Rightarrow A^c \cap B^c$
\end{example}
\begin{theorem}
\textbf{De Morgan's Law} sates for any events $A$ and $B$
\begin{outline}[enumerate]
\1 $(A \cup B)^c = A^c \cap B^c$
\1 $(A \cap B)^c = A^c \cup B^c$
\end{outline}
\end{theorem}
\begin{example}
Exactly one of $A$ and $B$ is denoted as $$A\cup B \cap (A \cup B)^c = A \cup B \cap (A^c \cup B^c)$$
\end{example}
More Definitions:
\begin{outline}[enumerate]
\1 \textbf{Disjoint(mutually exclusive)}
\2 $A\cap B = {} = \varnothing$
\1 $A$ implies $B$
\2 $A \subset B$
\2 $A \cap B = A$
\end{outline}
\newpage
%------------------------------------------------
\subsection{Probability Function}
\subsubsection{Definition}
\begin{definition}
Probability func $P$ defined on a \underline{finite} sample space $\Omega$ assigns each event $A \in \Omega$ a number $P(A)$ s.t.
\begin{outline}[enumerate]
\1 $P(A)\geq 0$
\1 $P(\Omega) = 1$
\1 $P(A \cup B) = P(A) + P(B)$
\2 if $A$ and $B$ disjoint.
\end{outline}
where $P(A)$ is the probability that event $A$ occurs.
\end{definition}
\subsubsection{Calculating by Counting}
Calculating by counting only applies when
\begin{outline}[enumerate]
\1 All outcomes of $\Omega$ are equally likely
\2 $\Omega$ is finite
\end{outline}
Then,
$$P(A) = \frac{\text{number of outcomes belonging to } A}{\text{ Total number of outcomes in }\Omega}$$
%\begin{theorem}
%    This is a theorem.
%\end{theorem}
%
%\begin{proposition}
%    This is a proposition.
%\end{proposition}

%\begin{principle}
%    This is a principle.
%\end{principle}

% Maybe I need to add one more part: Examples.
% Set style and colour later.


\newpage
%-------------------------
\chapter{Summation of Definitions and theorems}
\section{Definitions}
\begin{outline}[enumerate]
\1 \textbf{Event}
\2 Subset of Sample Space
\2 Relation between events
\3 Intersect
\4 denoted $A \cap B$
\3 Union
\4 denoted $A \cup B$
\3 Complement
\4 denoted $A^c$
\1 \textbf{Event}
\2 Subset of sample space
\1 \textbf{(Random) Experiment}
\2 Mechanism/Phenomenon that results in random or unpredictable outcomes
\1 \textbf{Probability}
\2 Numeric  Value of certainty/uncertainty
\1 \textbf{Sample Space}
\2 Set of all possible outcomes (from experiment)
\2 Denoted $\Omega$
\end{outline}
\section{Theorems}
% ------------------------------------------------------------------------------
% Reference and Cited Works
% ------------------------------------------------------------------------------

\bibliographystyle{IEEEtran}
\bibliography{References.bib}

% ------------------------------------------------------------------------------

\end{document}
